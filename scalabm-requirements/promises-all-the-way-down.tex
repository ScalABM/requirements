\documentclass[a4paper]{article}
\usepackage{graphics}
%%\usepackage[english,greek]{babel}
\setlength{\parindent}{0mm}
\setlength{\parskip}{1.8mm}
\newtheorem{definition}{Definition}
\newtheorem{theorem}{Theorem}
\newtheorem{proof}{Proof}
\title{It's promises (almost) all the way down...}
\date{\today}
\author{Daniel Tang, David Pugh, J. Doyne Famer}
%%\linespread{1.3}

\begin{document}
%%\selectlanguage{english}
\maketitle


\section{Motivation}
Roughly speaking our work is motivated by two key ideas:\footnote{
%
Don't forget to cite work by John G. and various papers by Kiyotaki and Moore (possibly others as well)!
%
} 
\begin{enumerate}
	\item ``All economic agents are banks'' - Hyman Minksy
	\item ``All interactions between banks involve a swap of IOUs'' - Perry Mehrling
\end{enumerate}
We have designed our low-level API around a slight translation of the above ideas:
\begin{enumerate}
	\item each economic \texttt{Actor} is a ``bank'';
	\item each interaction between economic \texttt{Actors} involves a \texttt{transfer} of a \texttt{Promise}.
\end{enumerate}

This immediately raises the following questions:
\begin{enumerate}
	\item What do we mean by an economic \texttt{Actor}?
	\item What we we mean when we say that all economic \texttt{Actors} are "banks"?
	\item What is a \texttt{Promise}?
	\item What does it mean to \texttt{transfer} a \texttt{Promise}?
\end{enumerate}

\section{Actor}
In this section we define the low-level API for an economic \texttt{Actor} and define what it means to say that each economic \texttt{Actor} is a ``bank.''

\begin{enumerate}
	\item each actor has a balance sheet containing assets and liabilities;
	\item each actor faces a sequence of cash-flow constraints.
\end{enumerate}

\section{In the beginning...}

\subsection{Defining a \texttt{Good}}

\subsection{\texttt{Actions} over \texttt{Goods}}

\section{Promises, promises...}
Borrowed title of this section from 1997 paper by John G. John G. defines an asset as a promise together with some collateral.

\subsection{Definition of a \texttt{Promise}}
A \texttt{Promise} represents a commitment between a \texttt{promisor} and some \texttt{promisee} to undertake certain \texttt{actions}, potentially involving both \texttt{goods} and additional \texttt{promises}, specified by some \texttt{Sentence} \texttt{when} a certain \texttt{StateofAffairs} has occurred. 

These considerations lead to the following low-level interface for the \texttt{Promise} trait.

\begin{verbatim}
trait Promise {

  def promisor: ActorRef
 
  def promisee: ActorRef
  
  def actions: Sentence
  
  def when: StateOfAffairs

}
\end{verbatim}

Want to allow for the possibility that some promises can not be given (and therefore not transferred or exchanged) from one \texttt{Actor} to another \texttt{Actor}. To capture this feature we define a separate \texttt{GiveablePromise} trait.

\begin{verbatim}
trait GiveablePromise extends Promise {

  val isGiveable: Boolean

}
\end{verbatim}

\subsection{\texttt{Actions} over \texttt{Promises}}

\subsubsection{Uni-lateral \texttt{Actions}}
An \texttt{Actor} $i$ can \textit{unilaterally} decide to perform any of the following \texttt{Actions} with a \texttt{Promise}. 
\begin{itemize}
	\item \texttt{create} a new \texttt{Promise}: When an \texttt{Actor} creates a new \texttt{Promise}, the \texttt{Actor} becomes its \texttt{promisor} and the new \texttt{Promise} becomes a liability for that \texttt{Actor}. Similarly, the new \texttt{Promise} becomes an asset for which ever \texttt{Actor} is the \texttt{promisee}.
	\item \texttt{accept} a \texttt{Promise}: If an \texttt{Actor} $i$ chooses to \texttt{accept} a \texttt{Promise} from another \texttt{Actor} $j$, then the \texttt{Promise} is added as an asset to the balance sheet of \texttt{Actor} $i$ and as a liability to the balance sheet of \texttt{Actor} $j$.
	\item \texttt{reject} a \texttt{Promise}: An \texttt{Actor} $i$ can always choose to \texttt{reject} (i.e., \textit{not} \texttt{accept}) a \texttt{Promise} from another \texttt{Actor} $j$. Rejected promises are not added to balance sheets.
	\item \texttt{fulfill} a \texttt{Promise}: An \texttt{Actor} $i$ who is the \texttt{promisor} of a \texttt{Promise} may choose to perform \texttt{actions} specified in the \texttt{Sentence} \texttt{when} a certain \texttt{StateOfAffairs} has occurred. Once a \texttt{Promise} is successfully fulfilled, it is removed from both the balance sheet of its \texttt{promisor} and the balance sheet of its \texttt{promisee}.
	\item \texttt{break} a \texttt{Promise}: An \texttt{Actor} $i$ who is the \texttt{promisor} of a \texttt{Promise} may choose \textit{not} to perform \texttt{actions} specified in the \texttt{Sentence} \texttt{when} a certain \texttt{StateOfAffairs} has occurred. A decision to \texttt{break} a \texttt{Promise} is the same as a decision \textit{not} to \texttt{fulfill} a \texttt{Promise}.\footnote{
	%
	Breaking a \texttt{Promise} may or may not have consequences. Should consequences be specified in the original \texttt{Promise}? I think so.
	%
	} 
	\item \texttt{destroy} a \texttt{Promise}: An \texttt{Actor} $i$ who is the \texttt{promisee} of a \texttt{Promise} may choose to \texttt{destroy} that \texttt{Promise} prior to a certain \texttt{StateOfAffairs} occurring. Once destroyed a \texttt{Promise} is removed from both the balance sheet of the \texttt{promisee} \textit{and} the balance sheet of the \texttt{promisor}.   
	\item \texttt{give} a \texttt{Promise}: An \texttt{Actor} $i$ who is the \texttt{promisee} of a \texttt{Promise} may choose to \texttt{give} that \texttt{Promise} to another \texttt{Actor} $j$.
	\item \texttt{redeem} a \texttt{Promise}:  An \texttt{Actor} $i$ who is the \texttt{promisee} of a \texttt{Promise} may choose to \texttt{redeem} that \texttt{Promise} after a certain \texttt{StateOfAffairs} has occurred. Redemption of a \texttt{Promise} can be thought of as a request that the \texttt{promisor} of that \texttt{Promise} \texttt{fulfill} the \texttt{Promise}.
\end{itemize}

The above actions can be combined into a low-level, \texttt{PromiseMaker} trait that extends the \texttt{Actor} base trait:
\begin{verbatim}
trait PromiseMaker extends Actor {
  
  def create(promisee: ActorRef,
             actions: Sentence,
             when: StateOfAffairs): Promise
  
  def destroy(promise: Promise): Unit
  
  def accept(promise: Promise): Unit
  
  def reject(promise: Promise): Unit
  
  def fulfill(promise: Promise): Unit
  
  def break(promise: Promise): Unit
  
  def give(promise: Promise, other: ActorRef): Unit
  
  def redeem(promise: Promise): Unit
  
}
\end{verbatim}

\subsubsection{Cooperative \texttt{Actions}}
Two \texttt{Actors} $i$ and $j$ can cooperate \textit{bi-laterally} to perform additional actions over \texttt{Promises}:
\begin{itemize}
	\item \texttt{transfer} a \texttt{Promise}: An \texttt{Actor} $i$ who is the \texttt{promisee} of the \texttt{Promise} can \texttt{transfer} a \texttt{Promise} to \texttt{Actor} $j$ as follows:
	\begin{enumerate}
		\item \texttt{Actor} $i$ \texttt{give} \texttt{Promise} to \texttt{Actor} $j$
		\item \texttt{Actor} $j$ \texttt{accept} \texttt{Promise} from \texttt{Actor} $i$
	\end{enumerate}
	\item \texttt{exchange} of \texttt{Promises}: Two \texttt{Actors} $i$ and $j$ who are the \texttt{promisees} of different \texttt{Promises} can \texttt{exchange} these \texttt{Promises} with one another as follows:
	\begin{enumerate}
		\item \texttt{Actor} $j$ \texttt{create} new \texttt{Promise} \{\texttt{give} existing \texttt{Promise} to \texttt{Actor} $i$\}.
		\item \texttt{Actor} $j$ \texttt{transfer} new \texttt{Promise} to \texttt{Actor} $i$.
		\item \texttt{Actor} $i$ \texttt{transfer} existing \texttt{Promise} to \texttt{Actor} $j$
		\item \texttt{Actor} $j$ \texttt{fulfill} \texttt{Promise} to \texttt{Actor} $i$
	\end{enumerate}
\end{itemize}
A few things are worth noting about a bi-lateral \texttt{exchange}. First, by choosing to \texttt{fulfill} the new \texttt{Promise} in step 4, \texttt{Actor} $j$ gives his existing \texttt{Promise} to \texttt{Actor} $i$ (which completes the \texttt{exchange}). Second, the new \texttt{Promise} issued by \texttt{Actor} $j$ in step 1 involved a promise to \texttt{give} and \textit{not} a promise to \texttt{transfer} an existing  \texttt{Promise} to \texttt{Actor} $i$.\footnote{
%
Should it be possible to for an \texttt{Actor} to create a \texttt{Promise} that commits \textit{other} \texttt{Actors} to perform actions? Do we have any real world examples?.
%
}
Finally, note that an \texttt{exchange} could take place with \texttt{Actor} $i$ creating the new \texttt{Promise} in step 1. The important point is that either \texttt{Actor} $i$ or \texttt{Actor} $j$ (not necessarily both) must be able to \textit{credibly} commit to \texttt{give} its \texttt{Promise} upon receipt of the other's \texttt{Promise}.\footnote{
%
The credibility of any particular \texttt{Promise} should be endogenously determined within the model and \textit{not} imposed by us \textit{a priori}
%
}

It is interesting to compare the above bi-lateral exchange mechanism with a multi-lateral exchange mechanism involving cooperation between three \texttt{Actors} $i$, $j$, and $k$. Two \texttt{Actors} $i$, $j$ who are the \texttt{promisees} of different \texttt{Promises} can \texttt{exchange} these \texttt{Promises} using \texttt{Actor} $k$ as an intermediary as follows:
\begin{enumerate}
	\item \texttt{Actor} $k$ \texttt{create} new \texttt{Promise} \{\texttt{give} \texttt{Actor} $j$ \texttt{Promise} to \texttt{Actor} $i$\}.
	\item \texttt{Actor} $k$ \texttt{create} new \texttt{Promise} \{\texttt{give} \texttt{Actor} $i$ \texttt{Promise} to \texttt{Actor} $j$\}.
	\item \texttt{Actor} $k$ \texttt{transfer} new \texttt{Promise} to \texttt{Actor} $i$.
	\item \texttt{Actor} $k$ \texttt{transfer} new \texttt{Promise} to \texttt{Actor} $j$.
	\item \texttt{Actor} $i$ \texttt{transfer} existing \texttt{Promise} to \texttt{Actor} $k$
	\item \texttt{Actor} $j$ \texttt{transfer} existing \texttt{Promise} to \texttt{Actor} $k$
	\item \texttt{Actor} $k$ \texttt{fulfill} \texttt{Promise} to \texttt{Actor} $i$
	\item \texttt{Actor} $k$ \texttt{fulfill} \texttt{Promise} to \texttt{Actor} $j$
\end{enumerate}
An important feature of this multi-lateral process is that, so long as \texttt{Actor} $k$ can \textit{credibly} commit to \textit{both} \texttt{Actors} $i$ \textit{and} $j$, then the exchange between $i$ and $j$ can take place even if \textit{neither} \texttt{Actor} $i$ \textit{nor} \texttt{Actor} $j$ can bi-laterally commit to \texttt{give} its \texttt{Promise} upon receipt of the other's \texttt{Promise}.\footnote{
%
The difference between multi-lateral and bi-lateral commitment has been stressed by many monetary theorists, in particular Kiyotaki and Moore in a series of papers.
%
}
There are several interpretations of \texttt{Actor} $k$'s role in the above process.  One interpretation is that \texttt{Actor} $k$ is functioning as a central clearing party (CCP) for transactions between other \texttt{Actors}; another more institutional interpretation is that \texttt{Actor} $k$ is an actual \texttt{Market}.

One final cooperative action needs to be specified: \texttt{transfer} of a \texttt{Promise} that is liability for one \texttt{Actor} to some other \texttt{Actor}.  An \texttt{Actor} $i$ who is the \texttt{promisor} on a \texttt{Promise} can only \texttt{transfer} that \texttt{Promise} to another \texttt{Actor} $j$ with permission from the \texttt{promisee} of that \texttt{Promise}, \texttt{Actor} $k$. 
\begin{enumerate}
	\item \texttt{Actor} $i$ \texttt{create} new \texttt{Promise} \{\texttt{give} \texttt{Actor} $j$ existing \texttt{Promise} \}.
	\item \texttt{Actor} $k$ \texttt{accept} new \texttt{Promise}.
	\item \texttt{Actor} $i$ \texttt{fulfill} \texttt{Promise} to \texttt{Actor} $k$.
	\item \texttt{Actor} $j$ \texttt{accept} \texttt{Promise} from \texttt{Actor} $i$.
\end{enumerate}

\subsection{A language for \texttt{Promises}}
Having defined the concepts of a \texttt{Good} and a \texttt{Promise} as well as sets of \texttt{actions} over \texttt{Goods} and \texttt{Promises} that can be performed by an \texttt{Actor} or groups of \texttt{Actors} to complete the API we need to define a language (grammar?) for building \texttt{Sentences} that describe valid \texttt{Promises}. 

\subsubsection{Examples}
Need to build a catalogue of examples demonstrating how to build common contracts using our language.
%%\tableofcontents

%%\appendix

\section{\texttt{Markets} API}

\begin{enumerate}
	\item Must be able to add/remove routees from the router.
	\item Must be able to handle situations when market has no buyers (sellers).
\end{enumerate}

The low-level \texttt{PromiseMaker} API already incorporates markets in the sense that an individual \texttt{PromiseMaker} can act is an intermediary between other \texttt{PromiseMakers} in order to facilitate the exchange of \texttt{Promises}.  In this instance a \texttt{PromiseMaker} is acting as a broker or  ``market maker'' and is functionally equivalent to a market institution.  

However in order to streamline the implementation of market institutions in actual models we have designed a separate \texttt{Market} API on top of the low-level \texttt{PromiseMaker} API.  The design of the \texttt{Market} API is based around the notion that a market needs to have mechanisms for:
\begin{enumerate}
	\item matching \texttt{Promises} made by ``buyers'' with \texttt{Promises} made by ``sellers;''
	\item clearing (i.e., finding an appropriate price and quantity) for each matched set of \texttt{Promises}.
\end{enumerate}

\begin{verbatim}
trait Market extends PromiseMaker {

  val router: akka.routing.Router

}
\end{verbatim}

\subsection{Routing}
Note that each \texttt{Market} has a \texttt{Router}.

\texttt{MarketParticipants} can easily be added (removed) as \texttt{Routees} to a \texttt{Router} by sending \texttt{AddRoutee} or \texttt{RemoveRoutee} messages.

\subsubsection{Routing logic}

\subsection{TODO:}

\begin{itemize}
	\item Router will need to have a \texttt{DeathWatch} on all its \texttt{Routees}.
\end{itemize}

\subsection{\texttt{MarketParticipant} API}
Since not all \texttt{Actors} in the model will need to participate in markets, seems a good idea to have a separate \texttt{MarketParticipant} trait that must be implemented in order for an \texttt{Actor} to participate in market transactions.

\begin{verbatim}
trait MarketParticipant {
  ???
}
\end{verbatim}

We implement separate interfaces for each of these functions: a \texttt{MatchingEngine} interface for generating matched sets of \texttt{Promises} and a \texttt{ClearingEngine} interface for finding an appropriate price and quantity for each matched set of \texttt{Promises}. 
\begin{verbatim}
trait MatchingEngine {
  // details to be implemented
}

trait ClearingEngine {
  // details to be implemented
}
\end{verbatim}
Having separate interfaces for matching and clearing improves the clarity and maintainability of the code base while allowing for greater flexibility in the modeling of markets.


\end{document}
